\documentclass{article}
\usepackage[utf8]{inputenc}

\title{Elaboration I and II}
\author{Ellen Kirkpatrick }
\date{August 2019}

\begin{document}

\maketitle
\section{Research Area}
Great inequality exists in welfare outcomes across different population groups in Australia despite the implementation of the modern welfare system in the 1970s. To what extent does the welfare system embody the political tensions within the Australian context and has this led to the continued marginalisation and differentiation of these groups?

\section{Scoping problem}
A major problem in my research is being able to identify existing connections between sources from different databases based on key themes. The end goal is a program which can access multiple sources at once and highlight any common themes or terms across them. It would also be able to store metadata and annotations to make use of them in future more efficient

\section{Decomposition: finding and identifying connections between multiple qualitative sources}
\begin{itemize}
    \item Identify key themes and terms relevant to my research question. This may include welfare, Australia, inequality, politics, ideology.
    \item Enter these terms in Macquarie University library database and select multiple sources which appear to be relevant to these key terms. 
    \item Copy url details and commit these sources to  voyant-tools.org. 
    \item If Voyant is sufficient and is chosen as the program, use the trends and summary tools to identify recurring themes or patterns which are common across the different sources. 
    \item Use the reader tool on Voyant to read through abstracts to determine whether the source may be relevant for research project or a specific part of the research project.
    \item Once the relevance or topic area is determined, export the source and reference to zotero and test whether successful. 
    \item If Zotero works adequately, add annotations for each source.
    \item Create tags for each source on zotero according to key themes and to create connections between sources.
    \item Access sources from zotero for further reading, interpretation and qualitative analysis. 
\end{itemize}

\section{Elaboration I: planning }
\begin{itemize}
    \item Test whether Macquarie University library database provides the relevant metadata required for Harvard referencing system.
    \item If not successful, test another database. 
    \item Test whether Voyant is successful in identifying common themes or patterns within well known sources. 
    \item If Voyant successfully identifies themes or terms that are known to be in sources, test on sources which are new and unknown. 
    \item If Voyant is chosen as a useful online platform, test whether Zotero is able to extract metadata from sources in Voyant.
    \item If Voyant is not chosen, test whether Zotero can extract metadata from the university database. 
    \item Test whether metadata and source is successfully imported into Zotero.
    \item Find where source and metadata is stored.
    \item Test adding annotations to specific sources in Zotero.
    \item Find how to access annotations.
    \item Test whether tags can be added to multiple sources linking them together in Zotero.
    \item Test whether multiple sources can be searched and accessed at once according to their tag. 
\end{itemize}

\section{Elaboration II: testing and revision}
\textbf{Test 1: Macquarie University database}
\begin{itemize}
    \item Searched Macquarie University library database with key phrase: Australian welfare system. Chose 3 sources for test - a book chapter and 2 peer reviewed journal articles. 
    \item Checked that each of these sources could be accessed in full online and that they had adequate metadata for the requirements of the Harvard referencing system which could be exported. 
    \item All 3 sources were successfully accessed online and had the required metadata. The Macquarie University library database is an appropriate database that can be used for further research. 
    \end{itemize}
\textbf{Test 2: Reliability of Voyant}
    \begin{itemize}
    \item Opened Voyant in web browser to test whether it can successfully identify on common themes and identify links between sources. 
    \item Uploaded to Voyant 10 sources which are well known from using them in a semester 1 assignment. Submitted these to Voyant to test its reliability on identifying themes from well known sources.
    \item Voyant successfully identified 5 key themes which were directly related to the assignment and common to all sources. Voyant is reliable and can be used to determine relevance of sources.
    \end{itemize}
\textbf{Test 3: Transferring online sources to Voyant}
 \begin{itemize}
     \item Copied the 3 urls from the sources chosen from test 1 into Voyant.
     \item \textbf{Error:} Voyant picked up on the gateway information from Macquarie University rather than the actual sources. This test not successful.
     \item Suspended this test and changed approach. Returned to the original sources to proceed with testing Zotero, before returning to Voyant to test its usefulness in analysis.
 \end{itemize}
\textbf{Test 4: Zotero}
\begin{itemize}
    \item Opened Zotero and used the Mozilla Firefox icon to add 3 sources to 'My Library'.
    \item Checked that each source could be accessed online and if there was a pdf file available, it could also be accessed. Successful .
    \item Checked that the metadata for each source could be accessed and exported and that additional annotations could be made. Successful.
    \item Tested that tags could be created for sources so they could be linked together. Successful.
    \item Zotero is successful in test for storing sources, metadata and annotations. Sources can be linked through tags and all can be accessed by clicking on the source in the library it is stored in. Sources can be searched by title, author or by tag.
    \end{itemize}
\textbf{Test 5: Voyant II}
\begin{itemize}
    \item Uploaded sources to Voyant through Zotero. \textbf{Error-} same problem occured as test 3. Have not been successful in submitting sources to Voyant through copying urls. 
    \item Returned to sources from Macquarie University library database and downloaded them as .pdf files in a folder called 'test' on the Desktop. Uploaded the files to Voyant to see test whether there were any commonalities between them.
    \item Voyant identified common themes between texts and showed links between specific paragraphs on the trends tool. The contexts tool can link with Zotero for sources and metadata to be exported.
    \item Voyant passed test 5 and is useful for connecting sources and exporting in full to Zotero. 
\end{itemize}
\textbf{Summary:}\\
The tests in elaboration II successfully responded to the steps articulated in elaboration I. Test 3 was the only test to be unsuccessful in that sources could not be copied over to Voyant. This may be due to the fact these items have restricted access and were accessible through institutional access but not in general. The same sources could be used when downloaded first and uploaded in pdf format. \\
This creates an extra step in the process and is not so straightforward as outlined in the decomposition but Voyant was successful in test 5 in identifying common themes and being used to determine relevance. Zotero can also extract the sources and metadata straight from Voyant.\\
Elaboration II indicates that sources need to be downloaded first from the database, uploaded to Voyant and then extracted to Zotero where they can be linked through tags and additional annotations can be added. \\
In response to the initial scoping problem, these programs are helpful in connecting sources and storing them with metadata. 



\end{document}



